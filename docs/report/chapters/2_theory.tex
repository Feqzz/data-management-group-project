\chapter{Theoretical Background}
%Here you write about the background information related to technologies you used.

\begin{itemize}
\item \textbf{Semantic Web:} This is the web of linked data. The Semantic Web is made from web-enabled data stores, vocabularies, and rules for handling data \cite{semantic}.

\item \textbf{RDF:} Is a framework that describes resources. A resource can be anything, including documents, people, physical objects, and abstract concepts \cite{rdf}. Data that is structured in accordance with RDF, follows \textit{subject – predicate – object} triples \cite{rdf}.

\item \textbf{Ontology:} According to \cite{ontology}, an ontology is: \textit{'An ontology defines a common vocabulary for researchers who need to share information in a domain''}.

\item \textbf{TURTLE:} TURTLE is the textual representation of an RDF graph \cite{turtle}. The syntax is compact and natural and allows for an easy way to express the relationship between resources.

\item \textbf{SPARQL:} SPARQL is a query language for RDF \cite{sparql}. The syntax and semantics of SPARQL is defined by the fact that RDF is often being used for social networks, metadata, and personal information \cite{sparql}.

\item \textbf{The five stars:} We have previously mentioned that our objective includes getting a 5-star rating, but what does that mean? The five stars is a deployment scheme for LOD suggested by Tim Berners-Lee \cite{lod}. The scheme goes as follows \cite{lod}:

	\begin{enumerate}
	\item Data is published on the web in any format (png), with a license.
	\item Data is structured in a machine-readable format (xml).
	\item Data is structured in a non-proprietary format (csv).
	\item Data is structured as RDF (turtle, SPARQL).
	\item The data identifiers are linked to useful data sources.
	\end{enumerate}

\end{itemize}

