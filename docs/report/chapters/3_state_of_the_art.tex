\chapter{Literature Review}
%Try to do some literature review and find any similar system is existing. If yes, write how your project is different than them. If not, list similar work.\newline

% Her må vi nevne at Semicolon prosjektet har jobbet med norske linkede data. 
% Si at semicolon selv foreslår parkering

% Og evt andre tilbydere av parkeringsdata
% Si hvordan vi er bedre enn andre tilbydere av parkeringsdata, eks vegvesenet sin api. Hvordan vi forbedrer dataene og tilgjengelighet


\section{The Semicolon project}
The Semicolon project \cite{semicolon}, created by The Research Council of Norway was one of the major driving forces of linked open data in Norway. As the popularity of the Semantic Web dropped they lost their funding in 2014 \cite{semicolon-slutt}, but they have published a lot of research on the subject. Including an overview of existing Norwegian LOD, as well as proposed extensions and applications of both new and existing datasets. 

In \cite{semicolon-ii}, Semicolon propose applications that among other things will need access to parking lot data as LOD. As Statens Vegvesen (The Norwegian Public Roads Administration) provide all parking lots in Norway as a RESTful API, we found this to be a suitable dataset for our project, as it would enhance the usability of this data.


\section{The LOD cloud}
The Linked Open Data Cloud \cite{lod-cloud}  is maintained by the Insight Centre for Data Analytics and provide a graphical representation of linked open data available on the web. Every project linked to here can be considered as similar to ours. The way our project stands out (from most of them at least) is that our dataset is not static, and will need continous updates. Becuase of this we have set up an automatic workflow scheduler that updates our dataset periodically. The number of Norwegian contributions are also quite few, and there does not seem to exist any datasets containing parking lot information, increasing the novelty of our work.
