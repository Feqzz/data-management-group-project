\chapter{Discussion and Conclusions}
%Discuss your results, challenges and conclude your work here.

Even though we have made it possible to create complex SPARQL queries through our endpoint, they are still hard to write. With more time, we would have liked to explore the idea of an easy-to-use application that utilised our SPARQL endpoint. This would ensure that also non-programmers would be able to search for parking facilities.


Husk å nevne:
	"Old" and not very mature technologies
	Often made by researchers. Many stuff are deprecated, badly documented

	Hard to configure

	Everything is JAVA, we hate JAVA.

	Had to spend a lot of time researching

	Unfortunately the semantic web is already dead
	Not used in industry, not worth it for companies

	Difficult to understand how data is actually "published"


	However, on a positive note, if the dream of the semantic web will ever be realized, we will have made a meaningful contribution to it! ELLER NOE



The data from Statens vegvesen's API \cite{statensvegvesen} does not come complete. We had to create a script that queries their API over 400 times in order to get all the data that we need. We tried implementing our ontology with both an OWL and RDF based solution, before deciding on RDF as ???. The webpage went through several frameworks??\\

The duration of the project spanned for three weeks. During the first week, we sat together and researched technologies, solutions and the semantic web while also creating small proof-of-concept programs and a baseline. The following days we split the working tasks equally among all group members and began implementing our solution.
