\chapter*{Preface}
\addcontentsline{toc}{chapter}{Preface}

For our project in the subject Data Management (CS4010) at the University of South-Eastern Norway (USN) campus Kongsberg, we have been researching the Semantic Web with the intention of publishing our own dataset to the Linked Open Data (LOD) cloud.

\vspace{5mm}


We made the decision to make our project about the Semantic Web after having gone through the materials provided on the subject “Data on the web”. Whilst studying these materials, we came across an old article about LOD in Norway \cite{semicolon-ii}. At this point there were little doubt that this would be our project.

\vspace{5mm}

One of the first problems we faced during this project was the fact that it seems the Semantic Web is more or less dead in 2021. Most of the material we have found have been from 10 – 12 years ago. This made things more challenging than we initially anticipated. 

\vspace{5mm}

For acquiring and tranforming data, we have written software in Python, using suitable libraries to make it easy for us to produce data formatted as RDF. We also have made use of a lot of existing software like Apache Fuseki, LodView, YASGUI and Apache Airflow, for hosting and presenting the data.

\vspace{5mm}

Throughout the course of this project, we have learned a lot about semantic data and how it can be made, used and linked. We have gotten new insight into how one goes about finding, collecting, and formatting data in accordance with the  Resource Description Framework (RDF).

\vspace{5mm}

Lastly, we are quite happy with the final result. We feel we have learned something that can be applied in the real world despite the lack of industry interest in the Semantic Web. With this we mean that the individual tools and the midset we have picked up still hold value in its own right. In example can they still be used as part of a local intranet or as part of a data platform. And should the dream of the Semantic Web ever be fulfilled, we will have made a meaningful contribution to it.



